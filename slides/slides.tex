\documentclass[%handout,
	sans,
	12pt,
	%slidescentered,% center text on slide
	%draft,			% compile as draft version
	%notes,			% include nodes in slides
	%compress		% compress navigation bar
]{beamer}

\beamertemplatenavigationsymbolsempty

\usetheme{default}
\usecolortheme{orchid}
\setbeamertemplate{frametitle}
{
    \vspace*{1.5em}\insertframetitle\vspace*{-1.5em}
}
\setbeamertemplate{footline}[frame number]

\usepackage[T1]{fontenc}
\usepackage[utf8x]{inputenc}

\usepackage{mathpazo}
\usepackage[british]{babel}
\usepackage{csquotes}

\newcommand{\high}[1]{{\usebeamercolor[fg]{structure} #1}}
\newcommand{\bad}[1]{\textcolor{red}{#1}}
\newcommand{\gray}[1]{\textcolor{darkgray}{#1}}
\newcommand{\black}[1]{\textcolor{black}{#1}}

\usepackage{amsmath,amssymb}
\usepackage{upgreek}
\usepackage{booktabs}
\usepackage{hyperref}
\usepackage{graphicx}
\usepackage{colortbl}
\usepackage{url}
\usepackage{setspace}
\usepackage{wrapfig}
\usepackage{tabularx}
\usepackage{xspace}
\usepackage{mathpartir}

\usepackage{tikz}
\usetikzlibrary{trees, positioning}
\usetikzlibrary{shapes.geometric}


\newcommand{\NN}{\mathbb{N}}
\newcommand{\QQ}{\mathbb{Q}}
\newcommand{\RR}{\mathbb{R}}
\newcommand{\CC}{\mathbb{C}}
\renewcommand{\epsilon}{\varepsilon}
\renewcommand{\phi}{\varphi}
\def\braces#1{[#1]}
\newcommand{\wrt}{w.\,r.\,t.\xspace}
\newcommand{\eg}{e.\,g.\xspace}
\newcommand{\ie}{i.\,e.\xspace}
\DeclareMathOperator\caret{\char`\^}

\newcommand{\hastype}{\,:\,}
\newcommand{\cons}{::}
\newcommand{\corrto}{\overset{\scriptscriptstyle\wedge}{=}}
\newcommand{\listapp}{\mathbin{@}}
\newcommand{\listnil}{[\hskip0.3mm]}
\newcommand{\listnth}{\mathbin{!}}
\newcommand{\expectation}{\text{\upshape E}}

\usepackage{manfnt}
\newenvironment{danger}{\medbreak\noindent\hangindent=2pc\hangafter=-2%
  \clubpenalty=10000%
  \hbox to0pt{\hskip-\hangindent\hskip0.25em\raisebox{-0.25em}[0pt][0pt]{\dbend}\hfill}\small\ignorespaces}%
  {\medbreak\par}
  %\raisebox{-1.05em}[0pt][0pt]{\Huge\hskip.15em \stixdanger}

\newcommand{\etAl}{\textit{et al.}\xspace}

%\definecolor{mybg}{rgb}{0.9,0.9,0.9}
\definecolor{mybg}{rgb}{1,1,1}
\setbeamercolor{background canvas}{bg=mybg}

\title{Verification of an Approximation Algorithm for the Metric Travelling Salesperson Problem \vspace*{-0.5em}}
\author{\normalsize Fabian Hellauer}
\institute[]{\footnotesize Technische Universität München}
\date{\footnotesize16 Oktober 2019}

\begin{document}

\maketitle

\begin{frame}
\begin{center}
\includegraphics[width=5cm]{isabelle.pdf}
\end{center}
\end{frame}

\tableofcontents	%to-do: remove

\newcommand{\pivot}[1]{{\color{red}#1}}
\newcommand{\ltpiv}[1]{{\color{blue}#1}}
\newcommand{\gtpiv}[1]{{\color{olive}#1}}

\section{Verification of Algorithms}
\begin{frame}{Verification of Algorithms}
Reasons:
\begin{itemize}
	\item a basis for reliable software\pause
	\item find mistakes in algorithm explanations\pause
	\item prove that verification %of a handfull of pages
	is feasible (when using modern tools)\pause
	%this could be motivation to do more math theories, because it removes the need for experts to get them accepted.
\end{itemize}
Mathematical results might be a requirement.
%In this case here, from graph theory.
\end{frame}


\section{Tool assistance}
\begin{frame}{Tool assistance}% how Isabelle helps in verifying
Isabelle enforces great precision, but also helps:
\begin{itemize}
	\item automatic proofs for simple lemmas\pause %rather: for *trivial* statements
	\item information about the proof state, e.g.\ pending sub-proofs
	%at any point in a proof. (example: induction over natural numbers).
	%And once again, simple sub-goals can be solved automatically.
\end{itemize}
\end{frame}

\section{Approximation Algorithms}
% after the graphic: Since we work with undirected graphs, the reverse of a TSP tour is also a TSP tour (I omitted these dulicates in the graphic)
\begin{frame}
\begin{center}
\huge\high{Approximation Algorithms}%Isabelle's library of abstract algebra, where groups and rings are defined
\end{center}
\end{frame}

\subsection{Approximation Algorithms}
\begin{frame}{Approximation Algorithms}
	Why use HOL-Algebra?\pause
	
	\high{To formalise Galois theory!}\pause
	\begin{theorem}
		\upshape
		The polynomial $x^5 − x − 1$ has no root described by an expression of integers, $n$th roots and basic arithmetic operations.
	\end{theorem}\pause% we are interested in the field of rationals + the roots
	% type would depend on the constants. This is not possible in this logic.
	% One could model one particular type (the supertype), for different constants one would have to start again.
	% There is no quanitification over types
	\bad{I won't get that far!}
\end{frame}

\section{Field Extensions}
\begin{frame}
\begin{center}
	\huge\high{Field Extensions}
\end{center}
\end{frame}

\begin{frame}

A \emph{field extension} $L/K$ is a field $(\mkern-3mu\mid\mkern-6mu L,\otimes,1,\oplus,0\mkern-6mu\mid\mkern-3mu)$ where $K$ is a subset of $L$ and $(\mkern-3mu\mid\mkern-6mu K,\otimes,1,\oplus,0\mkern-6mu\mid\mkern-3mu)$ is again a field.
\begin{itemize}
\item $K/K$, $[K:K] = 1$\pause
%problem for a formalisation: K is often identified with its carrier set
\item $\RR/\QQ$, $[\RR:\QQ]=\infty$\pause
\item $\QQ(\sqrt{3})/\QQ$, $[\QQ(\sqrt{3}):\QQ] = 2$ %Field extensions like these are interesting when doing Galois theory.
\end{itemize}
\end{frame}

\section{Results} % what I contributed

\subsection{Vector Spaces}
\begin{frame}{Vector Spaces}\pause
$U \subseteq V$ is a \emph{subspace} iff it is a vector space with $V$'s operations.
%over the same field
\begin{theorem}
\upshape
Let $U \subseteq V$ be a subspace. Then:
\begin{itemize}
	\item $dim_K(U) \le dim_K(V)$.
	\item If $dim_K(U)=dim_K(V) < \infty$, then $U = V$.
\end{itemize} %needed basis extension theorem. As you see, many lemmas I needed were missing.
\end{theorem}\pause
\begin{theorem}
\upshape
Let $n := dim_K(V) < \infty$. Then $V \cong K^n$.
% Note that the map is not explicitely given.
\end{theorem}
\end{frame}

\subsection{Classification of simple algebraic extensions}
\subsubsection{Minimal Polynomial}
\begin{frame}{Minimal Polynomial}%We are in a context where we have fixed L/K and some α∈L
\textbf{definition} \textit{irr} \textbf{where}\\
\hskip1em\textit{irr = (ARG-MIN degree p.}\\
\hskip2em\textit{p ∈ carrier P ∧ monic p ∧ Eval p = $\textbf{0}_L$)}\pause%isabelle does not help in determining "L"
\\[4mm]
\textbf{context}\\
\hskip1em\textbf{assumes} \textit{algebraic}\\%explain algebraic
\textbf{begin}\\[2mm]\pause

(...38 lines later)\\[2mm]
\textbf{lemma} \textit{irr ∈ carrier P} \textbf{and} \textit{monic irr} \textbf{and} \textit{Eval irr = $\textbf{0}_L$}\\\pause
\textbf{and} \textit{irr ≠ \textbf{0}}\\\pause
\textbf{and} \textit{degree irr > 0}\pause
\\[3mm]
Uniqueness of the minimal polynomial takes another 180 lines. %But this is part of an actual proof, so this to be expected
%We keep that "algebraic" assumption for a bit...
\end{frame}

\begin{frame}{Minimal Polynomial II}
Next up:\\\pause
\textbf{lemma} \textit{ring-irreducible irr}\\\pause
\high{The library helps with the proof:}\hskip1em\pause :-)\\\pause%happened rarely
\hskip1em\textit{Eval} is a homomorphism $P → L$.\\\pause
\hskip1em$(irr)$ is the preimage of $\{\textbf{0}\}$ under $Eval$.\\\pause %The set of elements which Eval maps to zero
%Do I need to explain (irr)?
%other library facts tell us that (0) is a primeideal and ↓
\hskip1em Thus, $(irr)$ is a prime ideal.\\[4mm]\pause % The claim follows
(...)\\[4mm]\pause%Some other transformations
\begin{theorem}
	$P/(irr) \cong K(\alpha)$%If \alpha is algebraic
\end{theorem}%overall close to 400 lines because I had to define algebraic first
%There is actually quite a bit about such quotient structures
\end{frame}

\section{Problems}
\begin{frame}{Problems}\pause
\begin{itemize}
	\item missing lemmas, missing definitions\pause %that's how I got to do vectorspace proofs
	\item Bases are sets.\\\pause
	\bad{Sets do not have an ordering}
	\item Coefficients are functions.\\\pause
	\bad{need to prove membership in the correct function set at \emph{every} step.}
\end{itemize}
\end{frame}

\subsection{Encoding}
\begin{frame}{Encoding of $\infty$}%relevant?
Can we use type \emph{nat} for the degree of field extensions?\pause
\begin{itemize}
\item The degree is a vector space dimension.\pause%these *can* be zero (iff zvs)
\item However, it is always $\ge 1$.\pause
\item Thus, this encoding strips no information:\\%It could be adapted easily to use extended natural numbers
\textbf{definition} \textit{degree} \textbf{where}\\
\hskip1em\textit{degree = (if finite then vs.dim else 0)}
\end{itemize}
\end{frame}

\section{Example Instantiation}

\begin{frame}{Example Instantiation}%An easy lemma as example
An explicit description of simple extensions: %at least *more* explicit, still using set comprehension
%I have proven this, in fact also used it in the previous proof at the end, just skipped over it.
\[K(\alpha) = \{f(\alpha)/g(\alpha) | f,g \in P, g(\alpha) ≠ \textbf{$0_L$}\}\]\pause
An instance of this:\\
\[\RR(i) = \CC\]\pause
\textbf{definition} \textit{complex-field = (carrier = (UNIV\hastype complex set), monoid.mult = (*), one = 1, zero = 0, add = (+))}
\end{frame}

\begin{frame}

\textbf{lemma} generate\_field\_i\_UNIV:\\
\hskip1em\textit{generate-field complex-field (insert 𝗂 $\RR$) = UNIV}\\
\textbf{proof} -\\
\hskip1em\textbf{interpret} \textit{UP-field-extension complex-field ℝ P 𝗂}\\
\hskip2em(...)\\
\hskip1em \textbf{fix} \textit{x :: complex}\\
\hskip1em \textbf{let} \textit{?f = monom P (Im x) 1 $⊕_P$ monom P (Re x) 0}\\
\hskip1em \textbf{let} \textit{?g = monom P 1 0}\\[1mm]%no creativity. the term is too large when written like this
\hskip1em(...)\\[1mm]
\hskip1em \textbf{have} \textit{x = Eval ?f $\oslash_\text{complex-field}$ Eval ?g ∧ ?f ∈ carrier P\\\hskip2em
	∧ ?g ∈ carrier P ∧ Eval ?g ≠ $\textbf{0}_\text{complex-field}$}\\[1mm]
\hskip1em(...)\\[1mm]
\textbf{qed}% ~30 lines

\end{frame}

\section{Contributions}
\begin{frame}{Contributions}
\begin{itemize}
	\item Vector Spaces: 850 lines, 75 lines preliminaries\pause
	\item Field Extensions: 820 lines, 100 lines preliminaries\pause%hard to distinguish
	\item 400 lines in an old development\pause %annoying, mostly about subrings, which were not even defined when I started
	\item 120 lines example instantiations\pause %to test definitions
	\item improvements to the "VectorSpace" library\pause
	\item Documentation: 300 lines (= 10 pages)
	%everything counted once. About 550 changesets.
\end{itemize}
\end{frame}

\section{Conclusion}
\begin{frame}{Conclusion}
% partly stolen from Manuel
\begin{itemize}
\item Formalisation of textbook algebra is hard work when using Isabelle.\pause%but possible
\item Proofs get long due to a variety of reasons.\pause
\item Good libraries can help a lot.\pause\\[6mm] % as seen with the last result
%(If you don't like what you do, defer it to someone else)
\end{itemize}
Interesting project for HOL-Algebra:\\[2mm]
GaloisCVC by P.\ E.\ de Vilhena, M.\ Baillon and L.\ Paulson (ALEXANDRIA grant)
%so, I will keep an eye out on what happens to HOL-Algebra.
%If someone with push access wants to include my material, they can send me questions, I will try and answer them.
%The documentation is almost finished.
%Are there questions now?
\end{frame}

\section{Reference}
\begin{frame}{Reference}
	G.\ Kemper. Algebra. German lecture notes, 2018. \url{http://www-m11.ma.tum.de/fileadmin/w00bnb/www/people/kemper/lectureNotes/algebra\_nodates.pdf} 
\end{frame}

\end{document}
